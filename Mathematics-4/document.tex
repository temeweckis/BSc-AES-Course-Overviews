% Template setup and packages.
\documentclass[10pt,landscape,a4paper]{article}
\usepackage{multicol}
\usepackage{calc}
\usepackage{ifthen}
\usepackage{geometry}

% Custom packages.
\usepackage{amsmath}
\usepackage{mathtools}
\usepackage{amssymb}
\usepackage{mathrsfs}
\usepackage{stix2}
\usepackage{systeme}
\usepackage{graphicx}
\usepackage{float}
\usepackage{physics}
\usepackage{siunitx}
\usepackage{collcell} % loads array
\newcolumntype{M}{>{$} l <{$}}
\newcolumntype{U}{>{$[\collectcell\si} l <{\endcollectcell]$}}

% Hyperref. Remember to change title!
\usepackage{hyperref}
\hypersetup{pdfauthor={Teemu Weckroth},pdftitle={Mathematics 4}}

% Path to graphics folder.
\graphicspath{ {./figures/} }

% Change fonts for v and w.
\DeclareSymbolFont{txletters}{OML}{ntxmi}{m}{it}
\SetSymbolFont{txletters}{bold}{OML}{ntxmi}{b}{it}
\DeclareFontSubstitution{OML}{ntxmi}{m}{it}
\DeclareMathSymbol{v}{\mathalpha}{txletters}{`v}
\DeclareMathSymbol{w}{\mathalpha}{txletters}{`w}

% Must be below font changes to avoid errors.
\usepackage{bm}

% Commands for differentials. Redefines the underdot command!
\renewcommand\d{\mathop{}\!\mathrm{d}}
\newcommand\p{\mathop{}\!\mathrm{\partial}}
%\newcommand\e{\mathrm{e}}

% Commands for the set of real numbers and Lagrangian/Laplace.
\newcommand{\R}{\mathbb{R}}
\newcommand{\La}{\mathscr{L}}

% Unbreakable unit environment.
\newenvironment{absolutelynopagebreak}
{\par\nobreak\vfil\penalty0\vfilneg
	\vtop\bgroup}
{\par\xdef\tpd{\the\prevdepth}\egroup
	\prevdepth=\tpd}

% Shrink bullet points.
\renewcommand\labelitemi{$\vcenter{\hbox{\tiny$\bullet$}}$}

%
\ifthenelse{\lengthtest { \paperwidth = 11in}}
{ \geometry{top=.5in,left=.5in,right=.5in,bottom=.5in} }
{\ifthenelse{ \lengthtest{ \paperwidth = 297mm}}
	{\geometry{top=1cm,left=1cm,right=1cm,bottom=1cm} }
	{\geometry{top=1cm,left=1cm,right=1cm,bottom=1cm} }
}

% Turn off header and footer
\pagestyle{empty}

% Redefine section commands to use less space
\makeatletter
\renewcommand{\section}{\@startsection{section}{1}{0mm}%
	{-1ex plus -.5ex minus -.2ex}%
	{0.5ex plus .2ex}%x
	{\normalfont\large\bfseries}}
\renewcommand{\subsection}{\@startsection{subsection}{2}{0mm}%
	{-1explus -.5ex minus -.2ex}%
	{0.5ex plus .2ex}%
	{\normalfont\normalsize\bfseries}}
\renewcommand{\subsubsection}{\@startsection{subsubsection}{3}{0mm}%
	{-1ex plus -.5ex minus -.2ex}%
	{1ex plus .2ex}%
	{\normalfont\small\bfseries}}
\makeatother

% Define BibTeX command
\def\BibTeX{{\rm B\kern-.05em{\sc i\kern-.025em b}\kern-.08em
		T\kern-.1667em\lower.7ex\hbox{E}\kern-.125emX}}

% Don't print section numbers
\setcounter{secnumdepth}{0}

\setlength{\parindent}{0pt}
\setlength{\parskip}{0pt plus 0.5ex}

\begin{document}

\raggedright
\footnotesize
\begin{multicols}{3}
	
	
	% multicol parameters
	% These lengths are set only within the two main columns
	%\setlength{\columnseprule}{0.25pt}
	\setlength{\premulticols}{1pt}
	\setlength{\postmulticols}{1pt}
	\setlength{\multicolsep}{1pt}
	\setlength{\columnsep}{2pt}
	
	\part*{Mathematics 4: Differential Equations.}
	\begin{center}
		Teemu Weckroth, \today
	\end{center}
	
	\section{Linearity.}
	An $ n^{\text{th}} $-order ordinary differential equation for $ y(t) $ is called linear if it is of the form
	\[
		a_n(t) \frac{d^n y}{d t^n} + ... + a_2(t) \frac{d^2 y}{d t^2} + a_1(t) \frac{dy}{dt} + a_0(t) y(t) = b(t).
	\]
	And ODE with a term such as $ y^2 $ or $ y \frac{dy}{dt} $ is called non-linear.
	
	\section{Homogeneity.}
	The ODE is called homogeneous if $ b(t) = 0 $ and inhomogeneous if $ b(t) \neq 0 $.
	
	\section{Autonomy.}
	A differential equation is called autonomous when it does not depend explicitly on the independent variable ($ t $).\\
	An ODE with a term such as $ \sin (t)$ or $t \frac{dy}{dt} $ is called not autonomous.
	
	\section{Direction field.}
	A direction field for a single differential equation is a graph with:
	\begin{itemize}
		\item the independent variable on the horizontal axis;
		\item the dependent variable on the vertical axis;
		\item and arrows which indicate the rate of change of the dependent variable.
	\end{itemize}
	
	\section{Equilibrium point stability.}
	An equilibrium point is stable if any initial value close to the equilibrium point gives solutions that always remain close to the equilibrium point.\\
	Any equilibrium point which is not stable is called unstable.
	
	\section{The harmonic addition theorem.}
	\[
		c_1 \cos{\omega t} + c_2 \sin{\omega t} = R \cos{(\omega t - \delta)}
	\]
	\[
		R = \sqrt{c_1^2 + c_2^2}
	\]
	\[
		\delta = 
		\begin{cases}
			\arctan{(c_2/c_1)}       & ~\text{if $c_1$ > 0} \\
			\arctan{(c_2/c_1)} + \pi & ~\text{if $c_1 < 0$}
		\end{cases}
	\]
	
	\section{The Laplace Transform.}
	For a function $ f : [0, \infty) \rightarrow \R $:
	\[
		\La \{f\} (s) = F(s) = \int_{0}^{\infty} {e^{-st} f(t)~dt},
	\]
	provided the integral exists.
	
	\section{Linearity.}
	For constants $ a, b \in \R $:
	\[
		\La\{af(t) + bg(t)\} = a\La\{f(t)\} + b\La\{g(t)\}.
	\]
	
	\section{Shift property.}
	For constant $ a \in \R $:
	\[
		\La\{f(t)e^{at}\} = \La\{f(t)\}(s-a) = F(s-a)
	\]
	
	\section{Differentiation rule.}
	If $ \La\{f(t)\} = F(s) $, then $ \La\{f'(t)\} = sF(s) - f(0) $.\\
	Consequence:
	\begin{align*}
		\La\{f''(t)\} & = s(sF(s) - f(0)) - f'(0) \\
		              & = s^2F(s) - sf(0) - f'(0)
	\end{align*}
	
	\section{The Inverse Laplace Transform.}
	If $ f(t) $ is piecewise continuous and $ \La\{f\}(s) = F(s) $ exists, then $ f(t) $ is uniquely determined by $ F(s) $ by means of the Inverse Laplace Transform
	\[
		f(t) = \La^{-1}\{F(s)\}.
	\]
	
	\section{Solving linear ODE using Laplace Transforms.}
	Three steps:
	\begin{enumerate}
		\item Compute Laplace transform of the initial value problem.\\
		\item Compute solutions $ Y(s) $ in the $ s $-domain.\\
		\item Compute inverse Laplace transform to find solutions $ y(t) $ in the $ t $-domain.
	\end{enumerate}
	
	\section{Finding a partial fraction decomposition.}
	Five steps:
	\begin{enumerate}
		\item Factorize the denominator into linear and quadratic factors.\\
		\item For each factor $ (s-a)^k $ introduce $ k $ terms
		      \[
			      \frac{A_1}{s-a}, \frac{A_2}{(s-a)^2}, ..., \frac{A_k}{(s-a)^k},
		      \]
		      and for a factor $ ((s-a)^2 + b^2)^\ell$ add terms
		      \[
			      \frac{B_1 (s-a) + C_1}{(s-a)^2 + b^2}, \frac{B_2 (s-a) + C_2}{((s-a)^2 + b^2)^2}, ..., \frac{B_{\ell} (s-a) + C_{\ell}}{((s-a)^2 + b^2)^\ell}.
		      \]
		\item Put all the terms together into one fraction and compare with the given rational function.
		\item Comparing the numerators gives a system of $ n $ linear equations for the $ n $ unknown parameters.
		\item Solve the system and write down the partial fraction decomposition.
	\end{enumerate}
	
	\section{Step function.}
	The function
	\[
		u_a(t) = 
		\begin{cases}
			0, & ~\text{if $ 0 \leq t < a $} \\
			1, & ~\text{if $ t \geq a $}
		\end{cases}
	\]
	is called the Heaviside function or the unit step function.\\
	E.g.
	\[
		F(t) = 
		\begin{cases}
			0,   & ~\text{if $ t<2 $ or $ t \geq 4 $} \\
			t-2, & ~\text{if $ 2 \leq t < 4 $}
		\end{cases}
	\]
	$ F(t) $ in one line: $ F(t) = (t-2) \cdot (u_2(t) - u_4(t)) $.
	
	\section{The impulse function.}
	\[
		\text{Limit for}~\tau \rightarrow 0^+: d_{\tau}(t) = 
		\begin{cases}
			\frac{1}{2 \tau}, & -\tau < t < \tau \\
			0,                & \text{other}~t
		\end{cases}
	\]
	Dirac delta "function" $ \delta(t-a) $:
	\[
		\delta(t-a) =
		\begin{cases}
			0,      & \text{when}~t \neq a \\
			\infty, & \text{when}~t=a
		\end{cases}
	\]
	Properties:
	\begin{align*}
		\int_{-\infty}^{\infty} \delta(t-a)~dt     & =1    \\
		\int_{-\infty}^{\infty} f(t)\delta(t-a)~dt & =f(a)
	\end{align*}
	
	\section{Solving a system of ODE.}
	If $ \vb{u}_1(t) $ and $ \vb{u}_2(t) $ are two linearly independent solutions, then every solution can be written as $ \vb{u}(t) = c_1 \vb{u}_1 (t) + c_2 \vb{u}_2 (t) $.\\
	The function $ \vb{u}(t) = \vb{v} e^{rt} $ is a solution of $ \frac{d\vb{u}}{dt} = A\vb{u} $ precisely when $ \vb{v} $ is an eigenvector and $ r $ is an eigenvalue of $ A $.
	\begin{align*}
		 & \vb{u}_1(t) = \vb{v}_1 e^{r_1t},~\vb{u}_2(t) = \vb{v}_2 e^{r_2t}    \\
		 & \rightarrow \vb{u}(t) = c_1 \vb{v}_1e^{r_1 t} + c_2\vb{v}_2e^{r_2t}
	\end{align*}
	For inhomogeneous system, $ \vb{z}(t) = \vb{z}_c + \vb{z}_p $.\\
	Assume $ \vb{z}_p $ is of the form $ \vb{z}_p(t) = \vb{c} $ (constant).\\
	Solve $ \vb{c} $ from $ \frac{d\vb{c}}{dt} = A\vb{c} + \vb{b} $.
	
	\section{Rewrite to real form.}
	If $ \vb{u}_0 $ is a complex (non-trivial) solution of $ \vb{u}' = A\vb{u} $, then $ \Re(\vb{u}_0) $ and $ \Im(\vb{u}_0) $ are two linearly independent solutions.\\
	If $ \vb{u}_0(t) $ is a complex solution, then $ \vb{u}(t) = c_1 \Re(\vb{u}_0(t)) + c_2 \Im(\vb{u}_0(t)) $ is a real solution.
	
	\section{Classification of equilibrium points.}
	\[
		\frac{d}{dt} \begin{pmatrix}x\\y\end{pmatrix} = A \begin{pmatrix}x\\y\end{pmatrix}
	\]
	$ A $ is a constant $ 2 \times 2 $-matrix with eigenvalues $ r_1 $ and $ r_2 $ and corresponding eigenvectors $ \vb{v}_1 $ and $ \vb{v}_2 $.\\
	If $ r_1 $ and $ r_2 $ are real-valued, the solution is given by
	\[
		\begin{pmatrix}x(t)\\y(t)\end{pmatrix} = c_1 \vb{v}_1 e^{r_1 t} + c_2 \vb{v}_2 e^{r_2 t}.
	\]
	\begin{align*}
		r_1 \leq r_2 < 0 & \quad \text{stable node}            \\
		0 < r_1 \leq r_2 & \quad \text{unstable node}          \\
		r_1 < 0 < r_2    & \quad \text{saddle point, unstable}
	\end{align*}
	If $ r_1, r_2 = a \pm bi, (b \neq 0) $, the solution is given by
	\[
		\begin{pmatrix}x(t)\\y(t)\end{pmatrix} = \vb{c}_1 e^{at} \cos{bt} + \vb{c}_2 e^{at} \sin{bt}.
	\]
	\begin{align*}
		a < 0 & \quad \text{stable spiral point}   \\
		a > 0 & \quad \text{unstable spiral point} \\
		a = 0 & \quad \text{stable centre}
	\end{align*}
	
	\section{Phase portrait procedure}
	Four steps:
	\begin{enumerate}
		\item Calculate equilibrium points.\\
		\item Linearise ODE around equilibrium points.\\
		\item Characterise solutions near equilibrium points.\\
		\item Sketch a phase portrait.
	\end{enumerate}
	
	\section{Linearisation around equilibrium.}
	Deviation from equilibrium point $ \left(x_0,y_0\right) $: $ x=x_0+u, y=y_0+v $.
	\[
		\begin{cases}
			x' = F(x,y) \\
			y' = G(x,y)
		\end{cases}
		\rightarrow \quad
		\begin{cases}
			u' = F(x_0+u,y_0+v) \\
			v' = G(x_0+u,y_0+v)
		\end{cases}
	\]
	Linearisation:
	\[
		\begin{cases}
			u' \approx \frac{\partial F}{\partial x} \left(x_0,y_0\right)u + \frac{\partial F}{\partial y} \left(x_0,y_0\right)v \\
			v' \approx \frac{\partial G}{\partial x} \left(x_0,y_0\right)u + \frac{\partial G}{\partial y} \left(x_0,y_0\right)v
		\end{cases}
	\]
	
	\section{Linearisation.}
	Consider $ \frac{d}{dt} \begin{pmatrix}x\\y\end{pmatrix} = \begin{pmatrix}F(x,y)\\G(x,y)\end{pmatrix} $ and equilibrium point $ \left(x_0,y_0\right) $.\\
	The linearisation of an autonomous system at $ \left(x_0,y_0\right) $ is given by
	\begin{align*}
		\frac{d}{dt} \begin{pmatrix}u\\v\end{pmatrix} & =
		\begin{pmatrix}
			\frac{\partial F}{\partial x} \left(x_0,y_0\right) & \frac{\partial F}{\partial y} \left(x_0,y_0\right) \\
			\frac{\partial G}{\partial x} \left(x_0,y_0\right) & \frac{\partial G}{\partial y} \left(x_0,y_0\right)
		\end{pmatrix}\begin{pmatrix}u\\v\end{pmatrix} \\
		                                              & = J\left(x_0,y_0\right)\begin{pmatrix}u\\v\end{pmatrix},
	\end{align*}
	where $ \begin{pmatrix}u(t)\\v(t)\end{pmatrix} = \begin{pmatrix}x(t)-x_0\\y(t)-y_0\end{pmatrix} $.
	
	\section{Type change after linearisation.}
	\begin{tabular}{l l}
		\textbf{Linearised}                    & \textbf{Non-linear}                        \\
		\hline
		Unstable spiral point                  & Unstable spiral point                      \\
		Stable spiral point                    & Stable spiral point                        \\
		Stable node $ (r_1 < r_2 < 0) $        & Stable node                                \\
		\qquad\qquad\quad $ (r_1 = r_2 < 0) $  & Stable spiral point or stable node         \\
		Unstable node $ (0 < r_1 < r_2) $      & Unstable node                              \\
		\qquad\qquad\qquad $ (0 < r_1 = r_2) $ & Unstable spiral point or unstable node     \\
		Saddle point (unstable)                & Saddle point (unstable)                    \\
		Center point (stable)                  & (Un)stable spiral or center point (stable)
	\end{tabular}
	
	\section{Van der Pol equation.}
	\[
		u'' - \mu (1-u^2)u' + u = 0
	\]
	
	\section{Lorenz equations.}
	\begin{align*}
		\frac{dX}{dt} & = \sigma(-X+Y) \\
		\frac{dY}{dt} & = rX-Y-XZ      \\
		\frac{dZ}{dt} & = -bZ+XY
	\end{align*}
	
	\section{Separation of variables.}
	\[
		\text{PDE: } \frac{\partial u}{\partial t} = k\frac{\partial^2 u}{\partial x^2}
	\]
	Four steps:
	\begin{enumerate}
		\item Split $ u(x,t) $: $ u(x,t) = X(x) \cdot T(t) $.\\
		\item Substitute $ X(x)T(t) $ in PDE: $ \frac{\partial X(x)T(t)}{\partial t} = k\frac{\partial^2 X(x)T(t)}{\partial x^2}$ $\rightarrow X(x)\frac{dT(t)}{dt} = k\frac{d^2X(x)}{dx^2}T(t) \rightarrow XT' = kX''T $.\\
		\item Divide both side by $ kXT $: $ \frac{T'}{kT} = \frac{X''}{X} = \lambda $\\
		\item Rewrite as a set of ODE:
		      $ \begin{cases}
				      \frac{X''}{X} = \lambda \\
				      \frac{T'}{kT} = \lambda
			      \end{cases}
			      \rightarrow \quad
			      \begin{cases}
				      X'' - \lambda X = 0 \\
				      T' - k\lambda T = 0
			      \end{cases}
		      $
	\end{enumerate}
	
	\section{Hyperbolic cosine/sine.}
	\[
		\cosh{x} = \frac{e^x+e^{-x}}{2} \qquad \sinh{x} = \frac{e^x-e^{-x}}{2}
	\]
	
	\section{Non-trivial solutions of a boundary value problem.}
	\[
		y'' + \lambda y = 0,~y(0)=0,~y(\pi)=0,~\text{with $ \lambda $ a real number.}
	\]
	Find all numbers $ \lambda $ for which non-trivial solutions $ y(x) $ exist.\\
	Consider three cases: $ \lambda<0,~\lambda=0,~\lambda>0 $.\\
	\begin{align*}
		\lambda < 0: & ~\text{suppose}~\lambda=-\mu^2 (\mu>0),~\text{so}~y''-\mu^2y = 0. \\
		             & ~\text{Only trivial solution.}                                    \\
		\lambda = 0: & ~y'' = 0.                                                         \\
		             & ~\text{Only trivial solution.}                                    \\
		\lambda > 0: & ~\text{suppose}~\lambda=\mu^2 (\mu>0),~\text{so}~y''+\mu^2y = 0.  \\
		             & ~y(x)=c_2\sin(kx),~\text{with}~\lambda=\mu^2=k^2=1,4,9,16, ... .
	\end{align*}
	
	\section{A periodic function.}
	$ f(x) $ is periodic if there exists a $ T $ such that $ f(x + T) = f(x) $ for all $ x $.\\
	$ T $ is called a period of the function $ f(x) $.\\
	Convention: it is sufficient to consider the function on the interval $ \left[-L,L\right] $, where $ T=2L $.\\
	
	\section{Orthogonal functions.}
	For integers $ m $ and $ n $:
	\[
		\int_{-L}^{L} \cos\left(\frac{m\pi x}{L}\right) \cos\left(\frac{n\pi x}{L}\right)~dx = \begin{cases}0,~m\neq n\\L,~m=n\end{cases}
	\]
	\[
		\int_{-L}^{L} \cos\left(\frac{m\pi x}{L}\right) \sin\left(\frac{n\pi x}{L}\right)~dx = 0,~\text{for each}~m,n
	\]
	\[
		\int_{-L}^{L} \sin\left(\frac{m\pi x}{L}\right) \sin\left(\frac{n\pi x}{L}\right)~dx = \begin{cases}0,~m\neq n\\L,~m=n\end{cases}
	\]
	
	\section{Fourier series.}
	A periodic function $ f(x) $ with period $ T=2L $ can be expressed as a Fourier series:
	\[
		f(x) = c_0 \sum_{m=1}^{\infty} \left(a_m\cos\left(\frac{m\pi x}{L}\right) + b_m\sin\left(\frac{m\pi x}{L}\right)\right).
	\]
	
	\section{Calculating coefficients $ c_0, a_m,~\text{and}~b_m $.}
	\begin{align*}
		\text{Find $ c_0 $:} & ~\int_{-L}^{L}{f(x)~dx} = \int_{-L}^{L}{c_0~dx} = 2c_0L                                                                    \\
		\text{Find $ a_m $:} & ~\int_{-L}^{L}{f(x)\cos\left(\frac{m\pi x}{L}\right)~dx} = a_m\int_{-L}^{L}{\cos^2\left(\frac{m\pi x}{L}\right)~dx} = a_mL \\
		\text{Find $ b_m $:} & ~\int_{-L}^{L}{f(x)\sin\left(\frac{m\pi x}{L}\right)~dx} = b_m\int_{-L}^{L}{\sin^2\left(\frac{m\pi x}{L}\right)~dx} = b_mL
	\end{align*}
	
	\section{Even and odd function identities.}
	\[
		\text{Even function:}~f(x) = f(-x) \qquad \text{Odd function:}~f(x)=-f(-x)
	\]
	\begin{align*}
		\text{odd}+\text{odd}=\text{odd};      & ~~\text{even}+\text{even}=\text{even};     & \text{even}+\text{odd}=\text{neither} \\
		\text{odd}\cdot\text{odd}=\text{even}; & ~~\text{even}\cdot\text{even}=\text{even}; & \text{even}\cdot\text{odd}=\text{odd}
	\end{align*}
	\[
		\int_{-L}^{L}\text{odd}~dx=0 \qquad \int_{-L}^{L}\text{even}~dx=2\int_{0}^{L}\text{even}~dx
	\]
	
	\section{Fourier sine and cosine series.}
	Let $ f(x) $ be a function with $ T=2L $.\\
	\begin{align*}
		\text{Cosine series:}~f(x) & =\frac{a_0}{2} + \sum_{n=1}^{\infty} a_n\cos\left(\frac{n\pi x}{L}\right),~\text{where} \\
		~a_n                       & =\frac{2}{L} \int_{0}^{L} f_{\text{even}}(x) \cos\left(\frac{n\pi x}{L}\right)~dx.      \\
		\text{Sine series:}~f(x)   & =\sum_{n=1}^{\infty} b_n\sin\left(\frac{n\pi x}{L}\right),~\text{where}                 \\
		~b_n                       & =\frac{2}{L} \int_{0}^{L} f_{\text{odd}}(x) \sin\left(\frac{n\pi x}{L}\right)~dx.
	\end{align*}
	\[
		f_\text{even}(x) = \frac{1}{2}(f(x)+f(-x)) \qquad f_\text{odd}(x) = \frac{1}{2}(f(x)-f(-x))
	\]
	
	\section{Solving a partial differential equation.}
	Five steps:
	\begin{enumerate}
		\item Separation of variables.\\
		\item Solve ODE with two homogeneous conditions.\\
		\item Solve other ODE.\\
		\item Formulate general solution.\\
		\item Apply last condition.
	\end{enumerate}
	
	
	
	
	\newpage
	
	\part*{Various Differential Equations.}
	
	\section{Heat diffusion equation.}
	\[
		\alpha^2 \frac{\partial^2 u(x,t)}{\partial x^2} = \frac{\partial u(x,t)}{\partial t}
	\]
	
	\section{Wave equation.}
	\[
		a^2 \frac{\partial^2 u(x,t)}{\partial x^2} = \frac{\partial^2 u(x,t)}{\partial t^2}
	\]
	
	\section{Laplace equation.}
	\[
		\frac{\partial^2 u(x,y)}{\partial x^2} + \frac{\partial^2 u(x,y)}{\partial y^2} = 0
	\]
	
	\section{Filling a cistern.}
	When the floater moves up, the tap closes.\\
	Tap is completely closed when $ h = H $.
	\begin{tabular}{M |l U}
		t    & Time                    & \second            \\
		V(t) & Water volume in cistern & \cubic\deci\meter  \\
		A    & Area of the cistern     & \square\deci\meter
	\end{tabular}\\~\\
	Consider time interval $ [t, t+ \Delta t] $.\\
	Change in $ V $:
	\[
		\Delta V = V (t + \Delta t) -V(t).
	\]
	For $ \Delta V $ the following equation holds:
	\[
		\Delta V = c (H - h) \Delta t.
	\]
	We know that
	\[
		\Delta V = A \Delta h,
	\]
	where $ \Delta h = h(t + \Delta t) - h(t) $.\\
	Balance equation:
	\[
		A \Delta h = c(H - h) \Delta t.
	\]
	From the equation $ A \Delta h = c (H - h) \Delta t $, we can derive the ODE:
	\begin{align*}
		\frac{\Delta h}{\Delta t}                               & = \frac{c}{A} (H - h)                               \\
		\lim_{\Delta t \rightarrow 0} \frac{\Delta h}{\Delta t} & = \lim_{\Delta t \rightarrow 0} \frac{c}{A} (H - h) \\
		\frac{dh}{dt}                                           & = \frac{c}{A} (H - h)                               \\
		h(0)                                                    & = 0
	\end{align*}\\
	
	\section{Periodically forced oscillator.}
	\begin{align*}
		m \frac{d^2u}{dt^2} & = F_{\text{ext}} + F_{\text{friction}} + F_{\text{spring}} \\
		                    & = F_0 \cos{\omega t} - \gamma \frac{du}{dt} - ku
	\end{align*}
	\begin{tabular}{M |l}
		u      & Displacement of the oscillator     \\
		t      & Time                               \\
		m      & Mass of the oscillator             \\
		\gamma & friction coefficient $(>0)$        \\
		k      & Spring coefficient $(>0)$          \\
		F_0    & Amplitude driving force            \\
		\omega & Angular frequency of driving force
	\end{tabular}\\~\\
	\textbf{Solution without friction.}\\
	\[
		m u'' + ku = F_0 \cos{\omega t}
	\]
	\begin{align*}
		u(t) & = u_c(t) + u_p(t)                                                                                                                   \\
		     & = c_1 \cos{\omega_0 t} + c_2 \sin{\omega_0 t} + \left( \frac{\frac{F_0}{k}}{1 - \frac{\omega^2}{\omega_0^2}} \right) \cos{\omega t}
	\end{align*}\\
	Natural frequency $ \omega_0 = \sqrt{\frac{k}{m}} $.\\
	\textbf{Solutions with friction.}\\
	\[
		m u'' + \gamma u' + ku = F_0 \cos{\omega t}
	\]
	\begin{align*}
		u(t) & = u_c(t) + u_p(t)                      \\
		     & = u_c(t) + R \cos{(\omega t - \delta)}
	\end{align*}
	
	Amount of friction $ \Gamma = \frac{\gamma^2}{m k} $.\\
	Natural frequency $ \Omega_0 = \sqrt{\frac{k}{m} - \left(\frac{\gamma}{2 m}\right)^2} \approx \sqrt{\frac{k}{m}} = \omega_0 $, where $ 0 < \gamma < 2m\Omega_0 $.\\
	
	
	\section{System of ODE: mass-spring system.}
	\[
		my'' + ky' + cy = 0
	\]
	Take $ v = y' $.
	\[
		mv' + kv + cy = 0
	\]
	Define the derivatives.
	\[
		\begin{cases}
			y' = v \\
			v' = y'' = -\frac{c}{m}y - \frac{k}{m}v
		\end{cases}
	\]
	\[
		\frac{d}{dt} \begin{pmatrix} y\\ v\end{pmatrix} =
		\begin{pmatrix}
			0            & 1            \\
			-\frac{c}{m} & -\frac{k}{m}
		\end{pmatrix} \begin{pmatrix}y\\v\end{pmatrix}
	\]
	
	\newpage
\end{multicols}
\end{document}
