% Template setup and packages.
\documentclass[10pt,landscape,a4paper]{article}
\usepackage{multicol}
\usepackage{calc}
\usepackage{ifthen}
\usepackage{geometry}

% Custom packages.
\usepackage{amsmath}
\usepackage{mathtools}
\usepackage{amssymb}
\usepackage{mathrsfs}
\usepackage{stix2}
\usepackage{systeme}
\usepackage{graphicx}
\usepackage{float}
\usepackage{physics}
\usepackage{siunitx}
\usepackage{collcell} % loads array
\newcolumntype{M}{>{$} l <{$}}
\newcolumntype{U}{>{$[\collectcell\si} l <{\endcollectcell]$}}

% Hyperref. Remember to change title!
\usepackage{hyperref}
\hypersetup{pdfauthor={Teemu Weckroth},pdftitle={Mathematics 5}}

% Path to graphics folder.
\graphicspath{ {./figures/} }

% Change fonts for v and w.
\DeclareSymbolFont{txletters}{OML}{ntxmi}{m}{it}
\SetSymbolFont{txletters}{bold}{OML}{ntxmi}{b}{it}
\DeclareFontSubstitution{OML}{ntxmi}{m}{it}
\DeclareMathSymbol{v}{\mathalpha}{txletters}{`v}
\DeclareMathSymbol{w}{\mathalpha}{txletters}{`w}

% Must be below font changes to avoid errors.
\usepackage{bm}

% Commands for differentials. Redefines the underdot command!
\renewcommand\d{\mathop{}\!\mathrm{d}}
\newcommand\p{\mathop{}\!\mathrm{\partial}}
%\newcommand\e{\mathrm{e}}

% Commands for the set of real numbers and Lagrangian/Laplace.
\newcommand{\R}{\mathbb{R}}
\newcommand{\La}{\mathscr{L}}

% Unbreakable unit environment.
\newenvironment{absolutelynopagebreak}
{\par\nobreak\vfil\penalty0\vfilneg
	\vtop\bgroup}
{\par\xdef\tpd{\the\prevdepth}\egroup
	\prevdepth=\tpd}

% Shrink bullet points.
\renewcommand\labelitemi{$\vcenter{\hbox{\tiny$\bullet$}}$}

%
\ifthenelse{\lengthtest { \paperwidth = 11in}}
{ \geometry{top=.5in,left=.5in,right=.5in,bottom=.5in} }
{\ifthenelse{ \lengthtest{ \paperwidth = 297mm}}
	{\geometry{top=1cm,left=1cm,right=1cm,bottom=1cm} }
	{\geometry{top=1cm,left=1cm,right=1cm,bottom=1cm} }
}

% Turn off header and footer
\pagestyle{empty}

% Redefine section commands to use less space
\makeatletter
\renewcommand{\section}{\@startsection{section}{1}{0mm}%
	{-1ex plus -.5ex minus -.2ex}%
	{0.5ex plus .2ex}%x
	{\normalfont\large\bfseries}}
\renewcommand{\subsection}{\@startsection{subsection}{2}{0mm}%
	{-1explus -.5ex minus -.2ex}%
	{0.5ex plus .2ex}%
	{\normalfont\normalsize\bfseries}}
\renewcommand{\subsubsection}{\@startsection{subsubsection}{3}{0mm}%
	{-1ex plus -.5ex minus -.2ex}%
	{1ex plus .2ex}%
	{\normalfont\small\bfseries}}
\makeatother

% Define BibTeX command
\def\BibTeX{{\rm B\kern-.05em{\sc i\kern-.025em b}\kern-.08em
		T\kern-.1667em\lower.7ex\hbox{E}\kern-.125emX}}

% Don't print section numbers
\setcounter{secnumdepth}{0}

\setlength{\parindent}{0pt}
\setlength{\parskip}{0pt plus 0.5ex}

\begin{document}

\raggedright
\footnotesize
\begin{multicols}{3}
	
	% multicol parameters
	% These lengths are set only within the two main columns
	%\setlength{\columnseprule}{0.25pt}
	\setlength{\premulticols}{1pt}
	\setlength{\postmulticols}{1pt}
	\setlength{\multicolsep}{1pt}
	\setlength{\columnsep}{2pt}
	
	\part*{Mathematics 5.}
	\begin{center}
		Teemu Weckroth, \today.
	\end{center}
	
	\section{Landau's $\mathcal{O}$-symbol.}
	Let $ f $ and $ g $ be given functions. If there exists a positive $ r $ and a positive, finite $ M $ such that
	\[
		|f(x)| \leq M|g(x)|\ \text{for all} \ x\in [-r,r],
	\]
	we say that $ f(x) $ is of order $ g(x) $ for $ x $ going to zero.
	Symbolically we write this as
	\[
		f(x)=\mathcal{O}(g(x)) \ \text{for} \ x\rightarrow0.
	\]
	If $ f(x)=\mathcal{O}(x^p) $ and $ g(x)=\mathcal{O}(x^q) $ for $ x\rightarrow0 $ with $ p\geq0,q\geq0 $, then
	\begin{itemize}
		\item $ f(x)=\mathcal{O}(x^s) $ for $ x\rightarrow0 $ with $ 0\leq s\leq p $
		\item $ \alpha f(x)+\beta g(x)=\mathcal{O}(x^r) $ for $ x\rightarrow 0 $ with $ r=\min\{p,q\} $ for all $ \alpha,\beta\in\mathbb{R} $
		\item $ f(x)g(x)=\mathcal{O}(x^{p+q}) $ for $ x\rightarrow0 $
		\item $ \frac{f(x)}{x^t}=\mathcal{O}(x^{p-t}) $ for $ x\rightarrow0 $ if $ 0\leq t\leq p $
	\end{itemize}
	
	\section{Taylor expansions.}
	Assume $ f(x) \in C^{n+1}\,[a,b] $ is given. Then for all $ c,x\in(a,b) $ there exists a number $ \xi $ between $ c $ and $ x $ such that
	\[
		f(x)=T_n(x)+R_n(x),
	\]
	in which
	\[
		T_n(x)=\sum_{k=0}^{n}\frac{f^{(k)}(c)}{k!}(x-c)^k,
	\]
	and
	\[
		R_n(x)=\frac{f^{(n+1)}(\xi)}{(n+1)!}(x-c)^{n+1} = \mathcal{O}((x-c)^{n+1}).
	\]
	
	\section{Lagrange interpolation - linear.}
	Assume two data points $ (x_0,f(x_0)) $ and $ (x_1,f(x_1)) $ of a function $ f $ are given. The linear Lagrange basis polynomials $ L_{i1}, \ i\in\{0,1\} $ are defined by
	\[
		L_{01}=\frac{x-x_1}{x_0-x_1}\ \text{and}\ L_{11}=\frac{x-x_0}{x_1-x_0},
	\]
	and the linear Lagrange interpolation polynomial $ L_1 $ by
	\[
		L_1(x) = f(x_0)L_{01}(x)+f(x_1)L_{11}(x).
	\]
	Assume $ f\in C^2\,[a,b] $ with $ x_0,x_1\in[a,b] $. Then for each $ x\in[a,b] $ there exists a $ \xi\in(a,b) $ such that
	\[
		f(x)-L_1(x)=\frac{1}{2}(x-x_0)(x-x_1)f''(\xi).
	\]
	Furthermore it holds that
	\[
		|f(x)-L_1(x)|\leq\frac{1}{8}(x_1-x_0)^2\underset{\gamma\in(a,b)}{\max}|f''(\gamma)|.
	\]
	Assume two distinct measured data points $ (x_0,\hat{f}(x_0)) $ and $ (x_1,\hat{f}(x_1)) $ of a function $ f\in C^2\,[a,b] $ are given with $ x_0,x_1\in[a,b] $.
	Also assume $ |f(x_0)-\hat{f}(x_0)|\leq\varepsilon $ and $ |f(x_1)-\hat{f}(x_1)|\leq\epsilon $ and $ \hat{L}_1 $ is the linear Lagrange interpolation polynomial using the measured values.
	Then for all $ x\in[a,b] $ it holds that
	\[
		|L_1(x)-\hat{L}_1(x)|\leq\frac{|x_1-x|+|x-x_0|}{|x_1-x_0|}\varepsilon.
	\]
	
	\section{Lagrange interpolation - general.}
	Assume $ n+1 $ data points $ (x_0,f(x_0)) $, $ (x_1,f(x_1)) $, $ \ldots $, $ (x_n,f(x_n)) $ of a function $ f $ are given. The Lagrange basis polynomials $ L_{kn} $, $ k\in \{0,1,\ldots,n\} $ are defined by
	\[
		L_{kn} = \prod_{\substack{j=0\\j\neq k}}^{n} \frac{x-x_j}{x_k-x_j},
	\]
	and the $ n^\text{th} $ degree Lagrange interpolation polynomial $ L_n $ by
	\[
		L_n(x) = \sum_{k=0}^{n}f(x_k)L_{kn}(x).
	\]
	Assume $ f\in C^{n+1}\,[a,b] $ with $ x_0,\hdots,x_n\in[a,b] $.
	Then for each $ x\in[a,b] $ there exists a $ \xi\in(a,b) $ such that
	\[
		f(x)-L_n(x)=\frac{1}{(n+1)!}\prod_{k=0}^{n}(x-x_k)f^{(n+1)}(\xi).
	\]
	Furthermore it holds that
	\[
		|L_n(x)-\hat{L}_n(x)|\leq\sum_{k=0}^{n}|L_{kn}(x)|\varepsilon,
	\]
	with $ \varepsilon $ an upper bound for measurement errors.
	
	\section{Spline interpolation.}
	Consider nodes $ a=x_0<x_1<\cdots<x_n=b $ and a function $ f\in C\,[a,b] $. A spline of degree $ p $ is a piecewise polynomial $ s\in C^{p-1}\,[a,b] $ that assumes the values of $ f $ in the nodes, and is defined by
	\[
		s(x) =
		\begin{cases}
			s_0(x),     & x\in[x_0,x_1),     \\
			s_1(x),     & x\in[x_1,x_2),     \\
			\vdots      & \vdots             \\
			s_{n-1}(x), & x\in[x_{n-1},x_n],
		\end{cases}
	\]
	where $ s_k $, $ k\in\{0,1,\ldots,n-1\} $ are polynomials of degree $ p $.
	
	\section{Numerical integration.}
	Assume $ f\in C\,[x_L,x_R] $ and $ x_M = \frac{1}{2}(x_L+x_R) $. We can define the following integration rules:
	\begin{align*}
		\text{left rectangle rule}  & \quad\int_{x_L}^{x_R}f(x)\d x \approx (x_R-x_L)f(x_L)                     \\
		\text{right rectangle rule} & \quad\int_{x_L}^{x_R}f(x)\d x \approx (x_R-x_L)f(x_R)                     \\
		\text{midpoint rule}        & \quad\int_{x_L}^{x_R}f(x)\d x \approx (x_R-x_L)f(x_M)                     \\
		\text{trapezoidal rule}     & \quad\int_{x_L}^{x_R}f(x)\d x \approx \frac{1}{2}(x_R-x_L)(f(x_L)+f(x_R)) \\
		\text{Simpson's rule}       & \quad\int_{x_L}^{x_R}f(x)\d x                                             \\ &\quad \approx \frac{1}{6}(x_R-x_L)(f(x_L)+4f(x_M)+f(x_R))
	\end{align*}
	
	\section{Numerical integration - accuracy.}
	Assume $ f\in C^q \, [x_L,x_R] $, $ x_M=\frac{1}{2}(x_L+x_R) $ and $ m_q = \underset{x\in[x_L,x_R]}{\text{max}} \left|f^{(q)}(x)\right| $. Then it holds that:
	\begin{align*}
		 & \left| \int_{x_L}^{x_R}f(x)\d x - (x_R-x_L)f(x_L) \right|                             & \\
		 & \quad\leq\frac{1}{2} m_1 (x_R-x_L)^2                                                    \\
		 & \left| \int_{x_L}^{x_R}f(x)\d x - (x_R-x_L)f(x_R) \right|                             & \\
		 & \quad\leq\frac{1}{2} m_1 (x_R-x_L)^2                                                    \\
		 & \left| \int_{x_L}^{x_R}f(x)\d x - (x_R-x_L)f(x_M) \right|                             & \\
		 & \quad\leq\frac{1}{24} m_2 (x_R-x_L)^3                                                   \\
		 & \left| \int_{x_L}^{x_R}f(x)\d x - \frac{1}{2}(x_R-x_L)(f(x_L)+f(x_R)) \right|         & \\
		 & \quad\leq\frac{1}{12} m_2 (x_R-x_L)^3                                                   \\
		 & \left| \int_{x_L}^{x_R}f(x)\d x - \frac{1}{6}(x_R-x_L)(f(x_L)+4f(x_M)+f(x_R)) \right| & \\
		 & \quad\leq\frac{1}{2880} m_4 (x_R-x_L)^5
	\end{align*}
	
	\section{Composite numerical integration.}
	Assume $ f\in C\,[a,b] $ and $ a=x_0<x_1<\cdots<x_n=b $ are given nodes.\\
	If $ I_k $ is an integration rule that approximates $ \int_{x_{k-1}}^{x_k}f(x)\d x $, we can approximate $ \int_{a}^{b}f(x)\d x $ with the composite integration rule
	\[
		I = \sum_{k=1}^{n}I_k.
	\]
	
	\section{Composite numerical integration - accuracy.}
	Assume $ f $ is defined on $ [a,b] $ and $ a=x_0<x_1<\cdots<x_n=b $ with $ x_k=a+kh $ and $ h=\frac{b-a}{n} $.
	Also assume $ I $ is a composite integration rule based on the integration rule $ I_k $ for which we know $ |\int_{x_{k-1}}^{x_k}f(x)\d x-I_k|\leq c_kh^{p+1} $. Then
	\[
		\left|\int_{a}^{b}{f(x)\d x-I}\right|\leq c(b-a)h^p,
	\]
	where $ c=\text{max}\{c_1,c_2,\hdots,c_n\} $.
	
	\section{Composite numerical integration - errors.}
	Assume $ I $ is a composite integration rule based on the function $ f $ and $ \hat{I} $ is the same composite integration rule based on the function $ \hat{f} $ containing measurement and/or rounding error for which we know $ |f(x)-\hat{f}(x)|=\varepsilon(x)\leq\varepsilon_\text{max} $. Then
	\[
		\left|\int_{a}^{b}{f(x)\d x-\hat{I}}\right|\leq\left|\int_{a}^{b}{f(x)\d x}-I\right|+\left|I-\hat{I}\right|.
	\]
	
	\section{Initial-value problems.}
	A first-order initial-value problem for a scalar function $ y $ of the independent variable $ t $ is a system of the form
	\[
		\begin{cases}
			\begin{aligned}
				y'     & =f(t,y), & \text{for}\ t\geq t_0, \\
				y(t_0) & =y_0,    & 
			\end{aligned}
		\end{cases}
	\]
	where $ f $ is a known function and $ y_0 $ is a known value.
	
	\section{Initial-value problems - stability.}
	Consider the two first-order initial-value problems
	\[
		\begin{cases}
			\begin{aligned}
				y' & =f(t,y), & t\geq t_0, \\ y(t_0)&=y_0,&
			\end{aligned}
		\end{cases}
		\ \mkern-18mu\text{and} \
		\begin{cases}
			\begin{aligned}
				\tilde{y}' & =f(t,\tilde{y}), & t\geq t_0, \\ \tilde{y}(t_0)&=y_0+\varepsilon_0,&
			\end{aligned}
		\end{cases}
	\]
	where $ f $ is a known function and $ y_0 $ and $ \varepsilon_0 $ are known values. Define $ \varepsilon(t)=\tilde{y}(t)-y(t) $. The left initial-value problem is stable if
	\[
		|\varepsilon(t)| < \infty \ \text{for all} \ t\geq t_0,
	\]
	and absolutely stable if it is stable and
	\[
		\lim_{t\rightarrow\infty} |\varepsilon(t)|=0.
	\]
	Define $ \lambda=\pdv{f}{y}\/ (\hat{t},\hat{y}) $.
	The left initial-value problem is stable for typical value of $ \hat{t} $ and $ \hat{y} $ if $ \lambda\leq0 $ and absolutely stable if $ \lambda<0 $.
	
	\section{Test equation.}
	The differential equation
	\[
		y'=\lambda y
	\]
	is called the test equation.
	
	\section{Forward Euler method.}
	Consider the initial-value problem
	\[
		\begin{cases}
			\begin{aligned}
				y' & =f(t,y), & \text{for} \ t\geq t_0, \\ y(t_0)&=y_0.&
			\end{aligned}
		\end{cases}
	\]
	Define $ w_n $ as the numerical approximation of $ y(t_n) $. Then the Forward Euler method is defined by the discrete dynamical system
	\[
		\begin{cases}
			\begin{aligned}
				w_{n+1} & = w_n + \Delta tf(t_n,w_n), & \text{for} \ n\geq0, \\ w_0 &= y_0,&
			\end{aligned}
		\end{cases}
	\]
	where $ t_n = t_0 + n\Delta t $.
	\begin{flalign*}
		 & \text{Local truncation error} \ \tau_{n+1}=\mathcal{O}(\Delta t). &  \\
		%\text{Global truncation error} \ |e_n|&\leq\Delta t.&\\
		 & \text{Stability attained if} \ \Delta t\leq-\frac{2}{\lambda}.    & 
	\end{flalign*}
	
	\section{Backward Euler method.}
	Consider the initial-value problem
	\[
		\begin{cases}
			\begin{aligned}
				y' & =f(t,y), & \text{for} \ t\geq t_0, \\ y(t_0)&=y_0.&
			\end{aligned}
		\end{cases}
	\]
	Define $ w_n $ as the numerical approximation of $ y(t_n) $. Then the Backward Euler method is defined by the discrete dynamical system
	\[
		\begin{cases}
			\begin{aligned}
				w_{n+1} & = w_n + \Delta tf(t_{n+1},w_{n+1}), & \text{for} \ n\geq0, \\ w_0 &= y_0,&
			\end{aligned}
		\end{cases}
	\]
	where $ t_n = t_0 + n\Delta t $.
	\begin{flalign*}
		 & \text{Local truncation error} \ \tau_{n+1}=\mathcal{O}(\Delta t). &  \\
		%\text{Global truncation error} \ |e_n|&\leq\Delta t.&\\
		 & \text{Stability attained for} \ \text{all} \ \Delta t.            & 
	\end{flalign*}
	
	\section{Trapezoidal method.}
	Consider the initial-value problem
	\[
		\begin{cases}
			\begin{aligned}
				y' & =f(t,y), & \text{for} \ t\geq t_0, \\ y(t_0)&=y_0.&
			\end{aligned}
		\end{cases}
	\]
	Define $ w_n $ as the numerical approximation of $ y(t_n) $. Then the Trapezoidal method is defined by the discrete dynamical system
	\[
		\begin{cases}
			\begin{aligned}
				w_{n+1} & = w_n + \frac{1}{2}\Delta t(f(t_n,w_n)+f(t_{n+1},w_{n+1})), & \text{for} \ n\geq0, \\ w_0 &= y_0,& 
			\end{aligned}
		\end{cases}
	\]
	where $ t_n = t_0 + n\Delta t $.
	\begin{flalign*}
		 & \text{Local truncation error} \ \tau_{n+1}=\mathcal{O}(\Delta t^2). &  \\
		%\text{Global truncation error} \ |e_n|&\leq\Delta t^2.&\\
		 & \text{Stability attained for} \ \text{all} \ \Delta t.              & 
	\end{flalign*}
	
	\section{Modified Euler method.}
	Consider the initial-value problem
	\[
		\begin{cases}
			\begin{aligned}
				y' & =f(t,y), & \ \text{for} \ t\geq t_0, \\ y(t_0)&=y_0.&
			\end{aligned}
		\end{cases}
	\]
	Define $ w_n $ as the numerical approximation of $ y(t_n) $. Then the Modified Euler method is defined by the discrete dynamical system
	\[
		\begin{cases}
			\begin{aligned}
				\overline{w}_{n+1} & = w_n + \Delta tf(t_n,w_n),                                            & \text{for} \ n\geq0,  \\
				w_{n+1}            & = w_n + \frac{1}{2}\Delta t(f(t_n,w_n)+f(t_{n+1},\overline{w}_{n+1})), & \text{for} \ n\geq 0, \\
				w_0                & = y_0,                                                                 & 
			\end{aligned}
		\end{cases}
	\]
	where $ t_n=t_0+n\Delta t $.
	\begin{flalign*}
		 & \text{Local truncation error} \ \tau_{n+1}=\mathcal{O}(\Delta t^2). &  \\
		%\text{Global truncation error} \ |e_n|&\leq\Delta t^2.&\\
		 & \text{Stability attained if} \ \Delta t\leq-\frac{2}{\lambda}.      & 
	\end{flalign*}
	
	\section{RK4 method.}
	Consider the initial-value problem
	\[
		\begin{cases}
			\begin{aligned}
				y' & =f(t,y), & \text{for} \ t\geq t_0, \\ y(t_0)&=y_0.&
			\end{aligned}
		\end{cases}
	\]
	Define $ w_n $ as the numerical approximation of $ y(t_n) $. Then the Fourth-order method of Runge-Kutta (RK4 method) is defined by the discrete dynamical system
	\[
		\begin{cases}
			\begin{aligned}
				k_1     & = \Delta tf(t_n,w_n),                                         & \text{for} \ n\geq0, \\
				k_2     & = \Delta tf(t_n + \frac{1}{2}\Delta t, w_n+\frac{1}{2}k_1),   & \text{for} \ n\geq0, \\
				k_3     & = \Delta tf(t_n + \frac{1}{2}\Delta t, w_n + \frac{1}{2}k_2), & \text{for} \ n\geq0, \\
				k_4     & = \Delta tf(t_n+\Delta t, w_n+k_3),                           & \text{for} \ n\geq0, \\
				w_{n+1} & = w_n + \frac{1}{6}(k_1 + 2k_2 + 2k_3 + k_4),                 & \text{for} \ n\geq0,
			\end{aligned}
		\end{cases}
	\]
	where $ t_n=t_0+n\Delta t $ and $ w_0=y_0 $.
	\begin{flalign*}
		 & \text{Local truncation error} \ \tau_{n+1}=\mathcal{O}(\Delta t^4). &  \\
		%\text{Global truncation error} \ |e_n|&\leq\Delta t^4.&\\
		 & \text{Stability attained if} \ \Delta t\leq\frac{2.8}{|\lambda|}.   & 
	\end{flalign*}
	
	\section{Numerical stability.}
	Consider the two first-order initial-value problems
	\[
		\begin{cases}
			\begin{aligned}
				y' & =f(t,y), & t\geq t_0, \\ y(t_0)&=y_0,&
			\end{aligned}
		\end{cases}
		\ \mkern-18mu\text{and} \quad
		\begin{cases}
			\begin{aligned}
				\tilde{y}' & =f(t,\tilde{y}), & t\geq t_0, \\ \tilde{y}(t_0)&=y_0+\varepsilon_0,&
			\end{aligned}
		\end{cases}
	\]
	where $ f $ is a known function and $ y_0 $ and $ \varepsilon_0 $ are known values. Assume a time-integration method has been applied to both initial-value problems, leading to the numerical solutions $ w_n, n=0,1,\hdots $ and $ \tilde{w}_n, n=0,1,\hdots $.\\
	Define $ \overline{\varepsilon}_n = \tilde{w}_n-w_n $. The used time-integration method applied to the left initial-value problem is (numerically) stable if
	\[
		|\overline{\varepsilon}_n| < \infty \ \text{for all} \ n\geq0.
	\]
	
	\section{Amplification factor $ Q $.}
	The amplification factor $ Q $ of a time-integration method is defined by applying the time-integration method to the test equation. It satisfies the equation
	\[
		w_{n+1} = Q(\lambda\Delta t)w_n.
	\]
	\begin{align*}
		\text{forward Euler} \quad  & Q(\lambda\Delta t) = 1 + \lambda\Delta t                                                                                                     \\
		\text{backward Euler} \quad & Q(\lambda\Delta t) = \frac{1}{1-\lambda\Delta t}                                                                                             \\
		\text{trapezoidal} \quad    & Q(\lambda\Delta t) = \frac{1+\frac{1}{2}\lambda\Delta t}{1-\frac{1}{2}\lambda\Delta t}                                                       \\
		\text{modified Euler} \quad & Q(\lambda\Delta t) = 1 + \lambda\Delta t + \frac{1}{2}(\lambda\Delta t)^2                                                                    \\
		\text{RK4} \quad            & Q(\lambda\Delta t) = 1 + \lambda\Delta t + \frac{1}{2}(\lambda\Delta t)^2 + \frac{1}{6}(\lambda\Delta t)^3 + \frac{1}{24}(\lambda\Delta t)^4
	\end{align*}
	
	\section{Numerical stability - amplification factor.}
	Consider a time-integration method with amplification factor $ Q $ applied to the initial-value problem
	\[
		\begin{cases}
			\begin{aligned}
				y' & =f(t,y), & t\geq t_0, \\ y(t_0)&=y_0,&
			\end{aligned}
		\end{cases}
	\]
	and define $ \lambda = \frac{\partial f}{\partial y}(\hat{t},\hat{y}) $. The chosen time-integration method applied to the initial-value problem is stable if
	\[
		|Q(\lambda\Delta t)| \leq 1
	\]
	for typical values of $ \hat{t} $ and $ \hat{y} $.
	
	\section{Local truncation error $ \tau $.}
	Given is that $ y_{n+1}=y(t_{n+1}) $ is the exact solution and that $ z_{n+1} $ is the numerical approximation of $ y_{n+1} $ with a chosen numerical time-integration method with starting point $ y_n=y(t_n) $ and time step $ \Delta t $. The local truncation error at time step $ n+1 $, $ \tau_{n+1} $, is then defined as
	\[
		\tau_{n+1}=\frac{y_{n+1}-z_{n+1}}{\Delta t}.
	\]
	To obtain a formula for the local truncation error the following steps can be performed:
	\begin{enumerate}
		\item Formulate a Taylor expansion of $ y_{n+1} $ around $ \Delta t=0 $;
		\item Formulate a Taylor expansion of $ z_{n+1} $ around $ \Delta t=0 $;
		\item Substitute 1. and 2. into the definition;
		\item Simplify.
	\end{enumerate}
	Given is a time-integration method with amplification factor $ Q $. Then the local truncation error for this method applied to the test equation $ y'=\lambda y $ is given by
	\[
		\tau_{n+1}=\frac{e^{\lambda\Delta t}-Q(\lambda\Delta t)}{\Delta t}y(t_n).
	\]
	To obtain a formula for the local truncation error for the test equation the following steps can be performed:
	\begin{enumerate}
		\item Formulate a Taylor expansion of $ e^{\lambda\Delta t} $ around $ \Delta t=0 $;
		\item Formulate a Taylor expansion of $ Q(\lambda\Delta t) $ around $ \Delta t=0 $;
		\item Substitute 1. and 2. into the equation above;
		\item Simplify.
	\end{enumerate}
	
	\section{Global truncation error $ e $.}
	Given is that $ y_{n+1}=y(t_{n+1}) $ is the exact solution and that $ w_{n+1} $ is the numerical approximation of $ y_{n+1} $ with a chosen solution time-integration method with time step $ \Delta t $ after $ n+1 $ time steps. The global truncation error at time $ t_{n+1} $, $ e_{n+1} $, is then defined as
	\[
		e_{n+1}=y_{n+1}-w_{n+1}.
	\]
	
	\section{Consistency and convergence.}
	A numerical time-integration method is called consistent if
	\[
		\lim_{\Delta t\rightarrow0}\tau_{n+1}=0,
	\]
	where $ T=(n+1)\Delta t $ is a fixed time. If it is consistent and $ \tau_{n+1}=\mathcal{O}(\Delta t^p) $, we call the method consistent of order $ p $.\\
	A numerical time-integration method is called convergent if
	\[
		\lim_{\Delta t\rightarrow0}e_{n+1}=0,
	\]
	where $ T=(n+1)\Delta t $ is a fixed time. If it is convergent and $ e_{n+1}=\mathcal{O}(\Delta t^p) $, we call the method convergent of order $ p $.\\
	If a numerical method is both stable and consistent, it is convergent. In that case the global and local truncation errors are of the same order.
	
	\section{Error estimation.}
	Define $ w_N^{\Delta t} $ as the numerical approximation of $ y(t) $ using $ N $ time steps of size $ \Delta t $ and $ e(t,\Delta t)=y(t)-w_N^{\Delta t} $ as the global truncation error at time $ t $ using time steps of size $ \Delta t $.\\
	Consider a time-integration method that is convergent of order $ p $ and assume $ \Delta t $ is small enough.
	Then the global truncation error can be estimated by
	\[
		e(t,\Delta t)=c_p(t)\Delta t^p=\frac{w_N^{\Delta t}-w_{N/2}^{2\Delta t}}{2^p-1}.
	\]
	Consider a time-integration method that is convergent, but the order $ p $ is unknown and assume $ \Delta t $ is small enough.
	Then the order of the global truncation error can be estimated from
	\[
		2^p=\frac{w_{N/2}^{2\Delta t}-w_{N/4}^{4\Delta t}}{w_n^{\Delta t}-w_{N/2}^{2\Delta t}}.
	\]
	Assume $ Q(h) $ is an approximation of an unknown value $ M $.
	For sufficiently small $ h $ the truncation error $ M-Q(h) $ can be estimated by
	\[
		M-Q(h)=c_ph^p=\frac{Q(h)-Q(2h)}{2^p-1}
	\]
	if $ p $ is known. If $ p $ is not known, it can be estimated from
	\[
		2^p=\frac{Q(2h)-Q(4h)}{Q(h)-Q(2h)}.
	\]
	
	\section{Stability of systems of differential equations.}
	Consider a vector-valued initial-value problem
	\[
		\begin{cases}
			\begin{aligned}
				\textbf{y}'     & =A\textbf{y}+\textbf{g}(t), & t\geq t_0, \\
				\textbf{y}(t_0) & =\textbf{y}_0,              & 
			\end{aligned}
		\end{cases}
	\]
	with $ A $ an $ n\times n $-matrix. Define $ \lambda_j,j=1,2,\hdots,n $ as the eigenvalues of $ A $, repeated according to their algebraic multiplicity.
	The system is stable if
	\[
		\Re(\lambda_j)\leq0
	\]
	for $ j=1,2,\hdots,n $.\\
	Consider the vector-valued initial-value problem
	\[
		\begin{cases}
			\begin{aligned}
				\textbf{y}'     & =\textbf{f}(t,\textbf{y}), & t\geq t_0, \\
				\textbf{y}(t_0) & =\textbf{y}_0,             & 
			\end{aligned}
		\end{cases}
	\]
	with $ \textbf{y}\in\R^n $. Define $ \lambda_j,j=1,2,\hdots,n $ as the eigenvalues of the Jacobian matrix $ J(\hat{t},\hat{\textbf{y}}) $ of $ \textbf{f} $, repeated according to their algebraic multiplicity, and typical values of $ \hat{t} $ and $ \hat{\textbf{y}} $. The system is stable if
	\[
		\Re(\lambda_j)<0,
	\]
	for $ j=1,2,\hdots,n $.\\
	A numerical time-integration method with amplification factor $ Q $ is stable if
	\[
		|Q(\lambda_j\Delta t)|\leq 1,
	\]
	for $ j=1,2,\hdots,n $.
	
	\section{Stability regions.}
	Consider a numerical time-integration method with amplification factor $ Q $. The stability region $ S $ of this method is defined as
	\[
		S=\{\lambda\Delta t\in \mathbb{C}:|Q(\lambda\Delta t)|\leq 1\}.
	\]
	
	\section{Stiff systems.}
	Consider a solution $ \textbf{y}(t) $ of some initial-value problem, and assume it contains a rapidly-decaying part and a slowly-varying part. Then we call the rapidly-decaying part the transient of the solution, and the initial-value problem is called a stiff system.
	
	\section{Superstability.}
	Consider a numerical time-integration method with amplification factor $ Q $ applied to an $ n $-dimensional initial-value problem with eigenvalues $ \lambda_j, \ j=1,\hdots,n $. This method is superstable if it is stable and
	\[
		\lim_{\operatorname{Re}(\lambda_j\Delta t)\rightarrow-\infty}|Q(\lambda_j\Delta t)|<1
	\]
	for $ j=1,\hdots,n $.
	\begin{flalign*}
		 & \text{All explicit methods are never superstable.}      &  \\
		 & \text{The Backward Euler method is always superstable.} &  \\
		 & \text{The Trapezoidal method is never superstable.}     & 
	\end{flalign*}
	
	\section{First-order derivative.}
	Consider a function $ f\in C\,[a,b] $. Then the first-order derivative $ f' $ is defined by
	\[
		f'=\lim_{h\rightarrow0}\frac{f(x+h)-f(x)}{h},
	\]
	if this limit exists.
	
	\section{Finite differences.}
	Consider a function $ f\in C^1[a,b] $. The value $ f'(x) $ can be approximated by the following finite differences:
	\begin{align*}
		\text{forward difference} \quad Q_f(h)  & = \frac{f(x+h)-f(x)}{h},    \\
		\text{backward difference} \quad Q_b(h) & = \frac{f(x)-f(x-h)}{h},    \\
		\text{central difference} \quad Q_c(h)  & = \frac{f(x+h)-f(x-h)}{2h}. \\
	\end{align*}
	
	\section{Truncation error $ R $.}
	Consider a function $ f\in C^1[a,b] $ and a finite difference $ Q(h) $ as an approximation of $ f'(x) $. Then the truncation error $ R(h) $ is defined as
	\[
		R(h) = f'(x)-Q(h).
	\]
	Consider a function $ f\in C^3[a,b] $. Then:
	\begin{align*}
		R_f(h)=f'(x)-Q_f(h) & =-\frac{h}{2}f''(\xi)=\mathcal{O}(h)        & \ \text{with} \ \xi\in(x,x+h),     \\
		R_b(h)=f'(x)-Q_b(h) & =\frac{h}{2}f''(\eta)=\mathcal{O}(h)        & \ \text{with} \ \eta\in(x-h,x),    \\
		R_c(h)=f'(x)-Q_c(h) & =-\frac{h^2}{6}f'''(\zeta)=\mathcal{O}(h^2) & \ \text{with} \ \zeta\in(x-h,x+h).
	\end{align*}
	
	\section{Rounding error $ S $, total error $ E $.}
	Consider a function $ f\in C^1[a,b] $ and a finite difference $ Q(h) $ as an approximation of $ f'(x) $. Also assume $ \hat{f} $ is the function $ f $ including rounding errors and $ \hat{Q}(h) $ is the finite difference based on $ \hat{f} $. Then the rounding error $ S(h) $ is defined as
	\[
		S(h)=|Q(h)-\hat{Q}(h)|,
	\]
	and the total error $ E(h) $ is defined as
	\[
		E(h)=|f'(x)-\hat{Q}(h)|.
	\]
	
	\section{Boundary value problems.}
	A second-order boundary value problem is the (unknown) variable $ y(x) $ on $ [K,L] $ is a system of the form
	\[
		\begin{cases}
			\begin{aligned}
				-(p(x)y'(x))'+r(x)y'(x)+q(x)y(x) & =f(x), & K\leq x\leq L, \\
				a_Ky(K)+b_Ky'(K)                 & =c_K,  &                \\
				a_Ly(L)+b_Ly'(L)                 & =c_L,  & 
			\end{aligned}
		\end{cases}
	\]
	where the last two equations are called boundary conditions and all constants and functions are known. Assume $ p(x)>0 $ and $ q(x)\geq0 $.
	
	\section{Boundary conditions.}
	Assume a boundary condition of the form
	\[
		a_Ky(K)+b_Ky'(K)=c_K
	\]
	is given.
	\begin{flalign*}
		\text{If} \ a_K\neq0, \ b_K=0,    & \ \text{we call the BC a Dirichlet boundary condition.}      &  \\
		\text{If} \ a_K=0, \ b_K\neq0,    & \ \text{we call the BC a Neumann boundary condition.}        &  \\
		\text{If} \ a_K\neq0, \ b_K\neq0, & \ \text{we call the BC a Robin or mixed boundary condition.} & 
	\end{flalign*}
	
	\section{Finite difference method.}
	Assume a boundary-value problem in $ y(x) $ on $ [K,L] $ is given. The finite difference method constructs a system of the form
	\[
		A\textbf{w}=\textbf{f},
	\]
	where $ w_j $ is an approximation of $ y(x_j) $ and $ x_j=K+j\Delta x $ and $ (n+1)\Delta x=L-K $.\\
	Algorithm:
	\begin{enumerate}
		\item Define the grid size $ \Delta x=\frac{L-K}{n+1} $ and $ x_j=K+j\Delta x $.
		\item Evaluate the differential equation in $ x_j $ and replace all derivatives with finite differences.
		\item Approximate $ y(x_j) $ by $ w_j $ and ignore all remainder terms.
		\item Incorporate the boundary condition at $ x=K $.
		\item Incorporate the boundary condition at $ x=L $.
		\item Formulate $ A $, $ \textbf{w} $, and $ \textbf{f} $ such that $ \textbf{w} $ contains all unknown values and $ A\textbf{w}=\textbf{f} $.
		\item Solve $ A\textbf{w}=\textbf{f} $ for $ \textbf{w} $.
	\end{enumerate}
	
	\section{Discretisation - $ (p(x)y'(x))' $.}
	The term
	\[
		\left.(p(x)y'(x))'\right|_{x=x_j}
	\]
	is discretised with the $ \mathcal{O}(\Delta x^2) $ finite difference
	\[
		\frac{p(x_{j+1/2})(w_{j+1}-w_j)-p(x_{j-1/2})(w_j-w_{j-1})}{\Delta x^2}.
	\]
	
	\section{Discretisation - $ p(K)y'(K) $.}
	The term
	\[
		p(K)y'(K)
	\]
	is discretised with the $ \mathcal{O}(\Delta x^2) $ finite difference
	\[
		\frac{1}{2}\left(p(x_{-1/2})\frac{w_0-w_{-1}}{\Delta x}+p(x_{1/2})\frac{w_1-w_0}{\Delta x}\right),
	\]
	for the term coming from $ (p(x)y'(x))' $ and with the $ \mathcal{O}(\Delta x^2) $ finite difference
	\[
		p(x_0)\frac{w_1-w_{-1}}{2\Delta x}
	\]
	for the term coming from $ r(x)y'(x) $.
	
	\section{Discretisation - $ p(L)y'(L) $.}
	The term
	\[
		p(L)y'(L)
	\]
	is discretised with the $ \mathcal{O}(\Delta x^2) $ finite difference
	\[
		\frac{1}{2}\left(p(x_{n+1/2})\frac{w_{n+1}-w_n}{\Delta x}+p(x_{n+3/2})\frac{w_{n+2}-w_{n+1}}{\Delta x}\right),
	\]
	for the term coming from $ (p(x)y'(x))' $ and with the $ \mathcal{O}(\Delta x^2) $ finite difference
	\[
		p(x_{n+1})\frac{w_{n+2}-w_n}{2\Delta x}
	\]
	for the term coming from $ r(x)y'(x) $.
	
	\section{Upwind discretisations.}
	An upwind discretisation of the term $ vy'(x_j) $ in the differential equation $ -y''+vy'+q(x)y=f(x) $ is given by
	\[
		vy'(x_j)\approx
		\begin{cases}
			\begin{aligned}
				v\frac{w_j-w_{j-1}}{\Delta x}, & \ \text{if} \ v\geq0, &  \\
				v\frac{w_{j+1}-w_j}{\Delta x}, & \ \text{if} \ v<0.    & 
			\end{aligned}
		\end{cases}
	\]
	
	\section{Norms.}
	The Euclidean norm $ ||\textbf{x}|| $ of a vector $ \textbf{x}\in \R^n $ is defined as
	\[
		||\textbf{x}||=\sqrt{\sum_{i=1}^{n}x_i^2},
	\]
	where $ \textbf{x}=[x_1,x_2,\hdots,x_n]^T $.\\
	The natural matrix norm $ ||A|| $ of an $ n\times n $-matrix $ A $ related to the Euclidean norm is defined as
	\[
		||A||=\underset{||x||=1}{\text{max}}||A\textbf{x}||,
	\]
	where $ \textbf{x}\in\mathbb{R}^n $.
	
	\section{Gershgorin circle theorem.}
	The eigenvalues of an $ n\times n $-matrix $ A $ are located in the complex plane in the union of circles
	\[
		|z-a_{ii}|\leq\sum_{\substack{j=1\\j\neq i}}^{n}|a_{ij}|,
	\]
	where $ z\in\mathbb{C} $.
	
	\section{Condition number.}
	The condition number $ \kappa(A) $ of an $ n\times n $-matrix $ A $ is defined as
	\[
		\kappa(A)=||A||\cdot||A^{-1}||.
	\]
	If $ A $ is a symmetric $ n\times n $-matrix with eigenvalues $ \lambda_1,\lambda_2,\hdots,\lambda_n$, then
	\[
		\kappa(A)=\frac{|\lambda|_\text{max}}{|\lambda|_\text{min}}.
	\]
	
	\section{Local errors.}
	The local truncation error $ \boldsymbol{\varepsilon} $ of the finite difference method scheme $ A\textbf{w}=\textbf{f} $ is defined as
	\[
		\bm{\varepsilon}=A\textbf{y}-\textbf{f}
	\]
	where $ \textbf{y} $ contains the exact solutions at the same grid nodes as the approximations in $ \textbf{w} $.\\
	A finite difference method scheme $ A\textbf{w}=\textbf{f} $ is consistent if
	\[
		\lim_{\Delta x\rightarrow0}||\bm{\varepsilon}||=0.
	\]
	
	\section{Stability.}
	The finite difference method scheme $ A\textbf{w}=\textbf{f} $ is stable if there exists a constant $ C $ independent of $ \Delta x $ such that
	\[
		||A^{-1}||\leq C \ \text{for} \ x\rightarrow0.
	\]
	
	\section{Global errors.}
	The global truncation error $ \textbf{e} $ of the finite difference method scheme $ A\textbf{w}=\textbf{f} $ is defined as
	\[
		\textbf{e}=\textbf{y}-\textbf{w},
	\]
	where $ \textbf{y} $ contains the exact solutions at the same grid nodes as the approximations in $ \textbf{w} $.\\
	A finite difference method scheme $ A\textbf{w}=\textbf{f} $ is convergent if
	\[
		\lim_{\Delta x\rightarrow0}||\textbf{e}||=0.
	\]
	
	\section{Equations and solutions.}
	A solution $ p $ of an equation is of the form
	\[
		f(x)=0
	\]
	is called a root of the equation and a zero of the function $ f $.\\
	Assume $ f\in C\,[a,b] $ and $ f(x)f(y)<0 $ for some $ x\in[a,b] $ and some $ y\in[a,b] $. Then $ f $ has a zero in $ [a,b] $.\\
	A solution $ p $ of an equation of the form
	\[
		g(x)=x
	\]
	is called a fixed point of the function $ g $.\\
	Assume $ g\in C\,[a,b] $ and $ g(x)\in[a,b] $ for all $ x\in[a,b] $. Then $ g $ has a fixed point in $ [a,b] $.\\
	Assume $ g\in C^1\,[a,b] $ and that $ g $ has a fixed point in $ [a,b] $. If there is a positive value $ k<1 $ such that
	\[
		|g'(x)|\leq k \ \text{for all} \ x\in[a,b],
	\]
	then $ p $ is unique on $ [a,b] $.
	
	\section{Convergence.}
	Assume some numerical method generates a sequence $ \{p_n\}=p_0,p_1,p_2,\hdots $.
	If $ \lim_{n\rightarrow\infty}{p_n}=p $, we call the sequence convergent (to $ p $).
	If there exists $ \lambda>0 $ and $ \alpha>0 $ such that
	\[
		\lim_{n\rightarrow\infty}\frac{|p-p_{n+1}|}{|p-p_n|^\alpha}=\lambda,
	\]
	then $ \{p_n\} $ converges to $ p $ with order $ \alpha $ and asymptotic constant $ \lambda $.
	\begin{flalign*}
		\text{If} \ \alpha=1 & \ \text{the sequence is linearly convergent.}      &  \\
		\text{If} \ \alpha=2 & \ \text{the sequence is quadratically convergent.} & 
	\end{flalign*}
	Assume $ \{p_n\} $ satisfies
	\[
		|p-p_n|\leq k|p-p_{n-1}|, \ \text{for} \ n=1,2,\hdots
	\]
	with $ 0\leq k\leq1 $. Then $ \{p_n\} $ converges to $ p $.\\
	Stopping criteria (examples):
	\begin{itemize}
		\item $ |p-p_n|<\varepsilon $
		\item $ |p_n-p_{n-1}|<\varepsilon $
		\item $ \frac{|p_n-p_{n-1}|}{|p_n|}<\varepsilon $
		\item $ |f(p_n)|<\varepsilon $
	\end{itemize}
	
	\section{Bisection method.}
	Assume $ f $ is continuous on the interval $ [a,b] $ and $ f(a)f(b)<0 $.
	Then the Bisection method produce a sequence $ \{p_n\} $ that converges to a zero $ p\in[a,b] $ of $ f $ with order $ \alpha=1 $, asymptotic constant $ \lambda=\frac{1}{2} $, and
	\[
		|p-p_n|\leq\frac{b-a}{2^{n+1}}, \ \text{for} \ n=0,1,2,\hdots.
	\]
	Algorithm:
	\begin{enumerate}
		\item Set $ n=0, \ a_0=a, \ b_0=b $.
		\item For $ n=0,1,2,3,\hdots $:
		      \begin{enumerate}
			      \item $ p_n=\frac{1}{2}(a_n+b_n) $
			      \item If $ f(a_n)f(p_n)<0 $, set $ a_{n+1}=a_n, \ b_{n+1}=p_n $.
			      \item Otherwise set $ a_{n+1}=p_n, \ b_{n+1}=b_n $.
			      \item If converged, stop.
		      \end{enumerate}
	\end{enumerate}
	
	\section{Fixed-point iteration.}
	Assume $ g\in C\,[a,b] $.
	Then the fixed-point iteration produces a sequence $ \{p_n\} $ that could converge to a fixed point $ p $ of $ g $.\\
	Algorithm:
	\begin{enumerate}
		\item Set $ n=0 $ and pick $ p_0\in[a,b] $.
		\item For $ n=0,1,2,3,\hdots $:
		      \begin{enumerate}
			      \item $ p_{n+1}=g(p_n) $
			      \item If converged, stop.
		      \end{enumerate}
	\end{enumerate}
	Assume $ g\in C^1[a,b] $, $ g(x)\in[a,b] $ for all $ x\in[a,b] $ and that there is a positive value $ k<1 $ such that
	\[
		|g'(x)\leq k| \ \text{for all} \ x\in[a,b].
	\]
	Then the fixed-point iteration $ p_{n+1}=g(p_n) $ converges to a unique fixed point $ p\in[a,b] $ for each $ p_0\in[a,b] $.
	If $ |g'(p)|\neq0 $, then the fixed-point iteration converges to $ p $ with order $ \alpha=1 $ and asymptotic constant $ \lambda=|g'(p)| $. If $ |g'(p)|=0 $, then $ \alpha>1 $.
	
	\section{Newton-Raphson method.}
	Assume $ f\in C^1[a,b] $, a root $ \in[a,b] $ of $ f $ exists and $ f'(x)\neq0 $ on $ [a,b] $.
	Then the Newton-Raphson method produces a sequence $ \{p_n\} $ that could converge to a zero $ p $ of $ f $.\\
	Algorithm:
	\begin{enumerate}
		\item Set $ n=0 $ and choose $ p_0 $.
		\item For $ n=0,1,2,3,\hdots $:
		      \begin{enumerate}
			      \item $ p_{n+1}=p_n-\frac{f(p_n)}{f'(p_n)} $
			      \item If converged, stop.
		      \end{enumerate}
	\end{enumerate}
	Assume $ f\in C^2[a,b] $ and that $ f $ has a zero $ p\in[a,b] $ with $ f'(p)\neq0 $.
	Then there is a $ \delta>0 $ such that the Newton-Raphson method produces a sequence $ \{p_n\} $ that converges to $ p $ for any $ p_0\in[p-\delta,p+\delta] $ with order $ \alpha=2 $ and asymptotic constant $ \lambda=\left|\frac{f''(p)}{2f'(p)}\right| $.
	
	\section{Newton-Raphson method variants.}
	If $ f'(p_n) $ is unknown or unavailable in $ p_{n+1}=p_n-\frac{f(p_n)}{f'(p_n)} $ the following variants can be used:
	\begin{align*}
		\text{quasi-Newton method} \quad & f'(p_n)=\frac{f(p_n+h)-f(p_n)}{h} \ \text{with} \ h>0 \ \text{small} &  \\
		\text{secant method} \quad       & f'(p_n)=\frac{f(p_n)-f(p_{n-1})}{p_n-p_{n-1}}                        &  \\
		\text{Regula-Falsi method} \quad & \text{Use the Bisection method with the change}                      &  \\
		                                 & p_n=a_n-f(a_n)\frac{b_n-a_n}{f(b_n)-f(a_n)}                          & 
	\end{align*}
	
	\section{Systems of non-linear equations.}
	A system of non-linear equations is a system of the form
	\[\text{fixed-point form}\quad\begin{cases}
			g_1(x_1,\hdots,x_m) & =x_1,  \\
			g_2(x_1,\hdots,x_m) & =x_2,  \\
			                    & \vdots \\
			g_m(x_1,\hdots,x_m) & =x_m,
		\end{cases}\]
	or of the form
	\[\text{general form}\quad\begin{cases}
			f_1(x_1,\hdots,x_m) & =0,    \\
			f_2(x_1,\hdots,x_m) & =0,    \\
			                    & \vdots \\
			f_m(x_1,\hdots,x_m) & =0.
		\end{cases}\]
	In vector notation:
	\[
		\textbf{g}(\textbf{x})=\textbf{x} \ \text{or} \ \textbf{f}(\textbf{x})=\textbf{0}.
	\]
	
	\section{Fixed-point iteration for systems.}
	Assume $ \textbf{g}\in C(\mathbb{R}^m) $.
	Then the fixed-point iteration produces a sequence $ \{\textbf{p}^{(n)}\} $ that could converge to a fixed point $ \textbf{p} $ of $ \textbf{g} $.\\
	Algorithm:
	\begin{enumerate}
		\item Set $ n=0 $ and pick $ \textbf{p}^{(0)}\in\mathbb{R}^m $.
		\item For $ n=0,1,2,3,\hdots $:
		      \begin{enumerate}
			      \item $ \textbf{p}^{(n+1)}=\textbf{g}(\textbf{p}^{(n)}) $
			      \item If converged, stop.
		      \end{enumerate}
	\end{enumerate}
	
	\section{Newton-Raphson method for systems.}
	Assume $ \textbf{f}\in C(\mathbb{R}^m) $ and a root $ \textbf{p} $ of $ \textbf{f} $ exists.
	Then the Newton-Raphson method produces a sequence $ \{\textbf{p}^{(n)}\} $ that could converge to a zero $ \textbf{p} $ of $ \textbf{f} $.\\
	Algorithm:
	\begin{enumerate}
		\item Set $ n=0 $ and choose $ \textbf{p}^{(0)} $.
		\item For $ n=0,1,2,3,\hdots $:
		      \begin{enumerate}
			      \item Solve for $ \textbf{s}^{(n)} $: $ J(\textbf{p}^{(n)})\textbf{s}^{(n)}=-\textbf{f}(\textbf{p}^{(n)}) $.
			      \item $ \textbf{p}^{(n+1)}=\textbf{p}^{(n)}+\textbf{s}^{(n)} $
			      \item If converged, stop.
		      \end{enumerate}
	\end{enumerate}
	
	\section{Initial-and-boundary-value problems.}
	An initial-and-boundary-value problem in the function $ y=y(x,t) $ is a problem of the form
	%\[
	%	\begin{cases}
	%		\pdv{y}{t}=k(x,t)\pdv{x}(p(x,t)\pdv{y}{x})-r(x,t)\pdv{y}{x}-q(x,t)y+f(x,t),&\substack{L\leq x\leq K,\\t\geq t_0,}\\
	%		y(L,t)+p(L,t)\pdv{y}{x}(L,t)=c_L(t),&t\geq t_0,\\
	%		y(K,t)+p(K,t)\pdv{y}{x}(K,t)=c_K(t),&t\geq t_0,\\
	%		y(x,0)=y_0(x),&L\leq x\leq K.
	%	\end{cases}
	%\]
	\[
		\begin{cases}
			\begin{aligned}
				\pdv{y}{t}=k(x,t)\pdv{x}(p(x,t)\pdv{y}{x})-r(x,t)\pdv{y}{x}-q(x,t)y+f(x,t), & \ \scriptsize\begin{aligned}&L\leq x\leq K, \\ &t\geq t_0,\end{aligned} \\
				y(L,t)+p(L,t)\pdv{y}{x}(L,t)=c_L(t),                                        & \ t\geq t_0,                                                            \\
				y(K,t)+p(K,t)\pdv{y}{x}(K,t)=c_K(t),                                        & \ t\geq t_0,                                                            \\
				y(x,0)=y_0(x),                                                              & \ L\leq x\leq K.
			\end{aligned}
		\end{cases}
	\]
	
	\section{Semi-discretisation.}
	A semi-discretisation of an initial-and-boundary-value problem is a system of the form
	\[
		\begin{cases}
			\begin{aligned}
				\dv{u}{t}=K\textbf{u}+\textbf{r}(t), & \ t\geq t_0. \\
				\textbf{u}(t_0)=\textbf{y}_0,        & 
			\end{aligned}
		\end{cases}
	\]
	where $ \textbf{u}=\textbf{u}(t) $ is a vector containing the approximations $ u_i(t) $ of $ y(x_i,t) $.\\
	The semi-discretisation is obtained by applying the finite-difference method to the problem in the spatial coordinate.\\
	Any time-integration method above can be applied to the semi-discretisation. We use $ \textbf{w}^j $ as the approximation of $ \textbf{u}(t_j) $.
	The $ i $-th component of $ \textbf{w}^j $, $ w_i^j $, is an approximation of $ u_i(t_j) $ and therefore an approximation of $ y(x_i,t_j) $.\\
	The local truncation error is defined as
	\[
		\bm{\tau}^{j+1}=\frac{\textbf{y}^{j+1}-\textbf{z}^{j+1}}{\Delta t},
	\]
	where $ y_i^{j+1}=y(x_i,t_j) $ and $ \textbf{z}^{j+1} $ is the result of applying one time step with $ \textbf{w}^j=\textbf{y}^j $.
	
	\newpage
\end{multicols}
\end{document}
